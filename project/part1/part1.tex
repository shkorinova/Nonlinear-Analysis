\chapter{Нелинейные динамические системы}
В настоящее время теория нелинейных динамических систем (НДС) представляет собой эффективный математический аппарат, нацеленный на качественное и количественное исследование поведения реальной системы (натуры).
 
\section{Введение: фазовое пространство и его свойства}
\begin{definition}[Фазовое пространство (ФП)]
  \index{функция распределения!неизвестная}
  \index{выборка}
  Фазовое пространство --- это пространство состояний реальной системы; абстрактное математическое пространство, размерность которого определяется числом функций, необходимых для однозначного задания состояния реальной системы в некий момент времени.
\end{definition}

Функции, однозначно задающие состояние реальной системы в момент времени $t$:
\begin{equation*}
	x_i(t), \qquad i = \overline{1,n}.
\end{equation*}

Так как по осям координат ФП находятся значения $x_i(t)$, можно составить \textit{вектор состояния}:
\begin{equation*}
	\vec{x}_i(t) = (x_1(t), x_2(t), \ldots, x_n(t)).
\end{equation*}

Тогда эволюция состояния системы будет описана нелинейными (как правило) дифференциальными уравнениями (ДУ) $1$ порядка.
\begin{equation} \label{eq:evol_system_state}
	\frac{dx_i(t)}{dt} = f_i(x_1, x_2, \ldots, x_n, a_1, a_2, \ldots, a_k)
\end{equation}

или
\begin{equation*}
	\frac{d\vec{x}_i(t)}{dt} = \vec{F}(\vec{x}_i(t), t, \vec{a}).
\end{equation*}

где $a_j$ --- параметры системы, $j = \overline{1,k}$.

\begin{definition}[Фазовая точка]
	Фазовая (изображающая) точка --- это точка ФП, описывающая настоящее (в данный момент времени) состояние реальной системы.
\end{definition}

\begin{figure}[h!]
	\center\includestandalone[]{tikz/phaseSpaceTrajectory}
	\caption{Фазовая точка $x_0$ и траектория её движения.} 
	\label{fig:tikz:phaseSpaceTrajectory}
\end{figure}

(\ref{eq:evol_system_state}) --- уравнение движения фазовой точки.

\begin{definition}[Фазовая траектория (ФТр)]
	Траектория движения фазовой точки называется фазовой траекторией.
\end{definition}

Для решения (\ref{eq:evol_system_state}) нужно знать начальные условия, то есть координаты точки, из которой выходит фазовая траектория.

\begin{equation}\label{eq:start_condition}
	t=t_0: \qquad x_i(t_0)=x_0, \qquad i=\overline{1,n}
\end{equation}

или
\begin{equation*}
	\vec{x}_i(t_0)=\vec{x}_0.
\end{equation*}

Если система неподвижна, то фазовая траектория --- это фазовая точка.

\begin{definition}[Динамическая система (ДС)]
	Система называется динамической, если она описывается уравнением (\ref{eq:evol_system_state}) и задание (\ref{eq:start_condition}) полностью определяет её поведение в последующие моменты времени.
\end{definition}

\begin{remark}
	В геометрии ФП стоит отметить, что ФП с размерностью:
	\begin{itemize}
		\item $n=2k$ изучает симплектическая геометрия;
		\item $n=2k+1$ изучает контактная геометрия.
	\end{itemize}
\end{remark}

\begin{remark}
	Уравнения движения могут быть записаны и в старших производных. Для того, чтобы свести их к виду (\ref{eq:evol_system_state}) вводим замену переменных.
	\begin{eqnarray*}
		\frac{d^2x_i}{dt^2}=f_i(x_k) \qquad \textnormal{Замена:} \quad y_i=\frac{dx_i}{dt} \\
		\textnormal{Получаем систему:}
		\begin{cases}
			\frac{dy_i}{dt}=f_i(x_k) \\
			\frac{dx_i}{dt}=y_i
		\end{cases}
	\end{eqnarray*}
\end{remark}

\begin{remark}
	Число степеней свободы системы равно $\frac{n}{2}$, где $n$ --- размерность ФП.
\end{remark}

Более формально ДС принято определять следующим образом:

\begin{definition}[Динамическая система (ДС)]
	Пусть некоторый эволюционный оператор $\mathrm{T}^t$ преобразовывает некоторое начальное состояние $\mathrm{P}_0$ в момент времени $t_0$ в $\mathrm{P}(t)$ в момент времени $t$. \\
	$$\mathrm{T}^t: \mathrm{P}_0 \rightarrow \mathrm{P}(t).$$ \\
	Тогда под ДС понимается такая система, эволюционный оператор которой определяется соотношением:
	$$\mathrm{T}^t\mathrm{T}^\tau = \mathrm{T}^{t+\tau}.$$ \\
	То есть время аддитивно, а оператор мультипликативный. При этом подразумевается коммутируемость:
	$$\mathrm{T}^t\mathrm{T}^\tau = \mathrm{T}^\tau\mathrm{T}^t.$$
\end{definition}

Задание ДС является постановкой задачи Коши. Согласно теореме Коши при заданных (\ref{eq:start_condition}) решение (\ref{eq:evol_system_state}) существует и оно единственное. Это утверждение приводит к следующим следствиям:
\begin{itemize}
	\item ФТр не самопересекаются.
	\item Так как существует и обратная задача Коши (вводится замена $t=-t$), то пересечение двух и более ФТр также невозможно.
	\item Если малое шевеление не изменяет ФТр системы, то такая система --- \textit{грубая} и к ней применима теорема Коши. В реальности негрубых систем гораздо больше, чем грубых, и для них существует \textit{горизонт прогноза}. Примеры негрубых систем: изменение индекса Доу Джонса, погода.
	Также существуют системы только с вероятностными уравнениями, для которых даже невозможно записать уравнение движения (\ref{eq:evol_system_state}).
\end{itemize}

\begin{example}[Обычная ДС]
	\begin{eqnarray*}
		\ddot{x}(t) + b\dot{x}(t) + ax(t-\tau) = 0 \qquad 
		\textnormal{Замена:} \frac{dx}{dt}=y \\
		\textnormal{Получаем систему:}
		\begin{cases}
			\frac{dy}{dt}=-by-ax(t-\tau) \\
			\frac{dx}{dt}=y
		\end{cases}
	\end{eqnarray*}	
\end{example}

\begin{definition}[Фазовый портрет]
	Совокупность ФТр, соответствующих различным начальным условиям или значениям параметров называется фазовым портретом.
\end{definition}

Поведение ФТр дает информацию о поведении системы. Существуют методы, которые позволяют не решая систему уравнений, а исходя только из вида правой части (\ref{eq:evol_system_state}) получить представление о поведении ФТр системы. Это и есть задача качественной теории ДУ.

\begin{definition}[Фазовая плоскость (ФПл)]
	Фазовая плоскость --- это ФП с размерностью $2$.
\end{definition}

Уравнения, описывающие эволюцию ДС:
\begin{eqnarray}\label{eq:phase_sirface}
	\begin{cases}
		\frac{dx_1}{dt}=f_1(x_1, x_2, t) \\
		\frac{dx_2}{dt}=f_2(x_1, x_2, t).
	\end{cases}
\end{eqnarray}	

Уравнение ФТр получаем из (\ref{eq:phase_sirface}):
\begin{equation*}
	\frac{dx_1}{dx_2}=\frac{f_1(x_1, x_2, t)}{f_2(x_1, x_2, t)}.
\end{equation*}	

Разнообразие поведения ФТр на ФПл ограничено теоремой Бендиксона-Пуанкаре. ФТр может:
\begin{itemize}
	\item уйти на бесконечность;
	\item уткнуться в особую точку;
	\item намотаться на предельный цикл.
\end{itemize}

Особые точки, они же неподвижные точки или стационарные точки определяются из уравнения (\ref{eq:spec_point}).

\begin{equation}\label{eq:spec_point}
	f_i(x_1, x_2, \ldots, x_n)=0, \qquad i=\overline{1,n}.
\end{equation}	

\begin{definition}[Автономная ДС]
	Если в правой части (\ref{eq:evol_system_state}) при заданных значениях параметров время не входит в него явно, то такая ДС называется автономной.
	\begin{equation*}
		\frac{d\vec{x}}{dt}=\vec{F}(\vec{x}, \vec{a}).
	\end{equation*}
\end{definition}

\subsection{Особые точки и особые траектории}

Рассмотрим ФПл:
\begin{eqnarray*}
	\begin{cases}
		\frac{dx_1}{dt}=f_1(x_1, x_2) \\
		\frac{dx_2}{dt}=f_2(x_1, x_2).
	\end{cases}
\end{eqnarray*}

Уравнение ФТр:
\begin{equation*}
	\frac{dx_1}{dx_2}=\frac{f_1(x_1, x_2)}{f_2(x_1, x_2)}.
\end{equation*}

\begin{enumerate}
	\item Центр
	\item Фокус
	\item Узел
	\item Предельный цикл
	\item Седло
\end{enumerate}

В ФПл нет других особых точек (ОТ), кроме указанных, за исключением того случая, когда ОТ сближаются, а затем сливаются.

Анализ таких ситуаций относится к разделу современной математики, который называется теорией структурной устойчивости или \textit{теорией катастроф}.

\begin{definition}[Аттрактор]
	ОТ или области ФП любой размерности, притягивающие ФТр, называются аттракторами.
\end{definition}

\section{Теорема Лиувилля-Остроградского}

Рассмотрим в $n$-мерном ФП автономную ДС.
\begin{equation}\label{eq:moving}
	\frac{d\vec{x}}{dt}=\vec{f}(\vec{x}(t)),
\end{equation}

где $\vec{x}=(x_1, x_2, \ldots, x_n), \vec{f}=(f_1, f_2, \ldots, f_n)$.

Вблизи точки $\vec{x}$ элемент ФП равен:
\begin{equation*}
	d\Gamma=dx_1 dx_2 \cdot \ldots \cdot dx_n.
\end{equation*}

Пусть $D$ --- область ФП, тогда объём в этой области равен:
$$V =\int\limits_{D}d\Gamma.$$

Область $D$ образуют точки ФП, которые можно рассматривать как точки разных ФТр в каждый момент времени. Это \textit{фазовая капля} объёма $V$ в момент времени $t$.
\begin{align*}
	D &= D(t), \\
	V &= \int\limits_{D(t)}d\Gamma.
\end{align*}

\begin{theorem}[(Лиувилля-Остроградского)]\label{th:liouville_ostrograd}
	$$\frac{dV}{dt}=\int\limits_{D(t)}\mathrm{div}\vec{f}d\Gamma, \qquad \textnormal{где} \quad \vec{f}=\frac{d\vec{x}}{dt}.$$
\end{theorem}

\begin{proof}
	За время $\tau$ координаты точек области $D$ изменятся.
	$$x_i(t) \xrightarrow{\tau} x_i(t+\tau).$$
	
	Тогда и область $D$ и объём $V$ также изменятся соответственно.
	$$D(t) \xrightarrow{\tau} D(t+\tau).$$
	$$V(t) \xrightarrow{\tau} V(t+\tau).$$
	
	Учитывая тот факт, что ФТр не пересекаются, можно утверждать, что между $x_i(t)$ и $x_i(t+\tau)$ существует взаимно однозначное соответствие.
	$$x_i(t+\tau)=\phi(x_1, x_2, \ldots, x_n, \tau).$$
	
	Обозначим: $y_i=x_i(t+\tau)$, тогда $y_i=\phi(x_1, x_2, \ldots, x_n, \tau), \quad i=\overline{1,n}.$ Это соотношение можно интерпретировать как преорбразование системы координат $(x_1, x_2, \ldots, x_n)$ в систему координат $(y_1, y_2, \ldots, y_n)$. В новой системе координат элемент ФП равен:
	$$d\tilde{\Gamma}=dy_1 dy_2 \cdot \ldots \cdot dy_n.$$
	
	Два элемента ФП $d\Gamma$ и $d\tilde{\Gamma}$ связаны якобианом преобразования.
	$$d\tilde{\Gamma}=\mathrm{J} d\Gamma, \qquad \textnormal{где } \mathrm{J}=\abs[\Big]{\frac{\partial(y_1, y_2, \ldots, y_n)}{\partial(x_1, x_2, \ldots, x_n)}}.$$
	
	Получаем выражение для объёма:
	\begin{align}\label{eq:volume}
		V(t+\tau)=\int\limits_{D(t)}d\tilde{\Gamma}=\int\limits_{D(t)}\mathrm{J}d\Gamma.
	\end{align}
	
	Пусть $\tau$ достаточно малый интервал: $\tau=dt$.
	
	Используя уравнение движения (\ref{eq:moving}) в координатном виде можно записать следующее соотношение:
	$$y_i=x_i(t+\tau)=x_i(t+dt)=x_i(t)+dx_i=x_i(t)+f_idt.$$
	
	Тогда
	$$\mathrm{J}=\det\Big(\frac{\partial y_i}{\partial x_k}\Big)=
	\det\Big(\frac{\partial x_i}{\partial x_k}+dt\frac{\partial f_i}{\partial x_k}\Big)=\det(\hat{E}+\hat{F}dt),$$
	
	где $\hat{E}_{(n\times n)}$ --- единичная матрица, $\hat{F}=\Big(\frac{\partial f_i}{\partial x_k}\Big)$, $F_{ik}=\frac{\partial f_i}{\partial x_k}$, $(i,k)=\overline{1,n}$.
	
	Раскладываем $\mathrm{J}$ в ряд Тейлора по степеням $dt$ с точностью до бесконечно малого $2$ порядка:
	$$\det\Big(\frac{\partial y_i}{\partial x_k}\Big)=1+\mathrm{Tr} \Big(\frac{\partial f_i}{\partial x_k}\Big)dt+o((dt)^2), \qquad
	\textnormal{где } \mathrm{Tr}\Big(\frac{\partial f_i}{\partial x_k}\Big)=\mathrm{div}\vec{f}.$$
	
	Таким образом, якобиан равен:
	\begin{align}\label{eq:jacobian}
		\mathrm{J} \approx 1+\mathrm{div}\vec{f}dt.
	\end{align}
	
	Из выражений (\ref{eq:volume}) и (\ref{eq:jacobian}) получаем:
	$$V(t+\tau)=V(t)+dt\int\limits_{D(t)}\mathrm{div}\vec{f}d\Gamma \quad \Rightarrow \quad \frac{dV}{dt}=\int\limits_{D(t)}\mathrm{div}\vec{f}d\Gamma.$$
\end{proof}

\begin{consequence}
	Величина фазового объёма гамильтоновых систем не изменяется с течением времени.
	Рассмотрим гамильтонову систему:
	\begin{align}\label{eq:hamilton}
		\begin{cases}
			\frac{dP_i}{dt}=-\frac{\partial \mathcal{H}}{\partial x_i} \\
			\frac{dx_i}{dt}=\frac{\partial \mathcal{H}}{\partial P_i}.
		\end{cases}
	\end{align}
	
	Принято обозначение: $\mathrm{div}\vec{f}=\Lambda$. Тогда
	\begin{align*}
		\Lambda &=\sum\limits_{i=1}^{n}\Big(\frac{\partial \dot{x}_i}{\partial x_i} + \frac{\partial \dot{p}_i}{\partial p_i}\Big) =\Big[\textnormal{подставляем }(\ref{eq:hamilton})\Big] \\ &=\sum\limits_{i=1}^{n}\Big(\frac{\partial}{\partial x_i}\frac{\partial \mathcal{H}}{\partial P_i} + \frac{\partial}{\partial p_i}\frac{\partial \mathcal{H}}{\partial x_i}\Big)=0 \\
		\Lambda&=0 \quad \Rightarrow \quad \frac{dV}{dt}=0 \quad \Rightarrow \quad V=const. 
	\end{align*}
\end{consequence}

\begin{consequence}
	Пусть область $D$ достаточно мала так, чтобы подынтегральное выражение в (\ref{th:liouville_ostrograd}) можно было считать постоянным.
	\begin{align*}
		\frac{dV}{dt} &=V\mathrm{div}\vec{f} \\
		\frac{1}{V}\frac{dV}{dt} &=\mathrm{div}\vec{f}=\sum\limits_{i=1}^{n} \frac{\partial f_i}{\partial x_i}.
	\end{align*}
	
	Рассмотрим частный случай, когда $f_i$ --- линейная функция координат.
	\begin{align*}
		f_i &=\sum\limits_{k=1}^{n} a_{ik}x_k \quad \Rightarrow \quad
		\frac{1}{V}\frac{dV}{dt}=\sum\limits_{k=1}^{n} a_{ii}=\Lambda \quad \Rightarrow \\
		&\Rightarrow \quad V(t) = V_0\exp(\Lambda t), \quad \textnormal{где } V_0\equiv V(0).
	\end{align*}
\end{consequence}

\section{Математические модели и классификация ДС}

Существует две основные модели: поток и отображение.

\subsection{Модель-поток (flows)}

Модель задаётся ДУ:
\begin{equation}\label{eq:flows}
	\frac{dx_i}{dt}=f_i(\{x_k(t)\}, \{a_j\}, t), \qquad (i,k)=\overline{1,n}; j=\overline{1,m}.
\end{equation}

\begin{definition}[Модель-поток]
	Модель вида (\ref{eq:flows}), в которой время непрерывно, называется модель-поток.
\end{definition}
 
 Уравнение (\ref{eq:flows}) можно записать в векторном виде:
 \begin{equation}
	 \frac{d\vec{x}}{dt}=\vec{f}(\vec{x}(t), \vec{a}, t).
 \end{equation}
 
 Часто удобно считать правую часть уравнения (\ref{eq:flows}) не зависящей явно от времени $t$. Это всегда можно сделать введя динамическую переменную, численно совпадающую со временем.
 $$\frac{dx_{n+1}}{dt}=1 \quad \Rightarrow \quad x_{n+1}=t.$$
 
 В этом случае ФП является расширенным ФП.

\subsection{Модель-отображение (maps)}

Это класс моделей, в которых время дискретно: $\{t_n\}, n \in \mathbb{Z}$. Номер $n$ момента времени $t_n$ принимается за независимую переменную.

\begin{example}
	\begin{align*}
		t:& \qquad 0c \qquad 5c \qquad 10c \qquad 15c \qquad \ldots \\
		&\qquad \downarrow \qquad \downarrow \qquad \downarrow \qquad \downarrow \\
		n:& \qquad 1 \qquad \quad 2 \qquad \quad 3 \qquad \quad 4 \qquad \ldots
	\end{align*}
\end{example}

Тогда уравнения движения модели-отображения выражают значения динамических переменных в момент времени $(n+1)$ через значение динамических переменных в момент времени $n$.

\begin{equation}\label{eq:maps}
	\vec{X}(n+1)=\vec{M}(\vec{X}(n), \vec{a}).
\end{equation}

\begin{definition}[Модель-отображение]
	Модель вида (\ref{eq:maps}), в которой время дискретно, называется модель-отображение.
\end{definition}

Рассмотрим источники этих моделей. Пусть реальная система имеет характерный временной масштаб $\mathrm{T}$.
\begin{itemize}
	\item За состоянием потока $x(t)$ следят в избранные моменты времени $t_n$, а не в произвольные. Эти $t_n$ разделены интервалом $\tau \approx \mathrm{T}$.
	\begin{remark*}
		$\tau \leq \mathrm{T}$, но ни в коем случае не $\tau > \mathrm{T}$.
	\end{remark*}
	Таким образом,
	$$\vec{x}(t_0+n\tau)=\vec{x}_n \textnormal{ --- временной ряд}.$$
	
	\item Условие $\tau >> \mathrm{T}$ используется для численного решения уравнения (\ref{eq:flows}); потоки (\ref{eq:flows}) заменяем отображениями (\ref{eq:maps}), то есть дискретными разностными схемами для потоков.
	\begin{example}[Метод ломаных Эйлера]
		$$x_{n+1}=x_n+hf(x_n), \quad h << \mathrm{T}.$$
	\end{example}
	
	\item Источником модели-отображения являются также системы, в которых дискретный счет времени.
	\begin{example}[Динамика популяций]
		$u$ --- количество особей, \\
		$n$ --- номер поколения. \\
		$$u_{n+1}=u_n+u_{n-1}.$$
		ФТр этой системы с начальными условиями $u_1=1, u_2=1$ --- это числа Фибоначчи.
	\end{example}
\end{itemize}

\subsection{Классификация ДС}

\begin{definition}[Автономная ДС]
	Если ДС задана моделью потоков:
	\begin{equation}\label{eq:flows1}
	\frac{dx_i}{dt}=f_i(\{x_k(t)\}, \{a_j\}), \qquad (i,k)=\overline{1,n}; j=\overline{1,m},
	\end{equation}
	а в правую часть не входит время явно и все параметры не зависят от времени, то такая ДС называется автономной. 
\end{definition}

\begin{definition}[Неавтономная ДС]
	Если в правую часть входит явно время и существует такое $a_j$, зависящее от времени, то такая ДС называется неавтономной. 
\end{definition}

Данные определения справедливы и для модели-отображения, где вместо (\ref{eq:flows1}) используется уравнение:
\begin{equation}\label{eq:maps1}
	\vec{X}(n+1)=\vec{M}(\{\vec{X}(n)\}, \{\vec{a}\}).
\end{equation}

\begin{definition}[Локальная диссипация]
	Для ДС-потока локальной диссипацией $G(\vec{x})$ в точке $\vec{x}$ ФП принято называть дивергенцию поля фазовых скоростей в этой точке, взятой с обратным знаком.
	$$G(\vec{x})=-\mathrm{div}\dot{\vec{x}}=-\sum\limits_{i=1}^{n}\frac{\partial \dot{x_i}}{\partial x_i}=-\mathrm{div}\vec{f}.$$
\end{definition}

Из теоремы Лиувилля-Остроградского следует, что для малого $V$ вблизи точки $\vec{x}$ получаем уравнение:
$$\frac{1}{V} \frac{dV}{dt}=\mathrm{div}\vec{f} \qquad \Rightarrow \qquad \frac{1}{V} \frac{dV}{dt}=-G(\vec{x}).$$

Это означает, что относительная скорость изменения фазового объема равна диссипации с обратным знаком.

\begin{definition}[Диссипативная система]
	Системы, для которых $G(\vec{x})\neq 0$ во всех точках ФП называют диссипативными.
\end{definition}

\begin{definition}[Консервативная система]
	Системы, для которых $G(\vec{x})=0$ во всех точках ФП называют консервативными.
\end{definition}

\begin{remark}
	Согласно этому определению консервативность --- это сохранение фазового объема, а не энергии.
\end{remark}

Любые гамильтоновые системы являются консервативными, но все консервативные системы являются гамитоновыми.

\begin{example}
	\begin{equation}
		\begin{cases}
			\dot{x}=yz \\
			\dot{y}=xz \qquad \Rightarrow \qquad G(\vec{x})=0 \\
			\dot{z}=xy 
		\end{cases}
	\end{equation}
	
	Однако система не гамильтонова.
\end{example}

Рассмотрим диссипацию для системы , заданной моделью-отображением (\ref{eq:maps}). Для одного отображения:
$$V^{\prime}= \mathrm{J}V, \qquad \textnormal{где } \mathrm{J}=\det\Big(\frac{\partial M_i}{\partial x_i}\Big).$$

Найдем локальную диссипацию:
\begin{itemize}
	\item для потока:
	$$G(\vec{x})=-\frac{1}{V} \frac{dV}{dt}=-\frac{d\ln \abs{V}}{dt}.$$ \\
	\item для отображения:
	\begin{align*}
		G(\vec{x}) &=-\frac{\Delta \ln \abs{V}}{\Delta n}=\Big[\Delta n=1 \textnormal{ --- локальность} \Big]=-\Delta \ln \abs{V} \\
		&= -(\ln \abs{V^{\prime}}-\ln \abs{V})=-(\ln \abs{\mathrm{J}V}-\ln \abs{V})=-\ln \abs{\mathrm{J}}.
	\end{align*}
\end{itemize}

Таким образом, локальная диссипация для модели-отображения --- это логарифм абсолютной величины якобиана, взятый с обратным знаком.
$$G(\vec{x})=-\ln \abs{\mathrm{J}}.$$

\section{Основные типы задач теории ДС}

\begin{enumerate}
	\item Задача Коши
	\item Задача исследования устойчивости движения
	\item Задача исследования структуры ФП
	\item Задача исследования самой ДС
\end{enumerate}

\subsection{Задача Коши}

Пусть есть уравнение движения:
\begin{equation*}
	\frac{d\vec{x}}{dt}=\vec{f}(\vec{x}(t), \vec{a}).
\end{equation*}

Тогда требуется в заданный момент времени $t_0$, $\vec{x}(0)=\vec{x}_0$ найти её состояние в любой момент времени.

Решение задачи Коши должно удовлетворять прямой и обратной задачи Коши (прямая: $t \rightarrow +\infty$, обратная: $t \rightarrow -\infty$).

\subsection{Задача исследования устойчивости движения}

Устойчивость определяет качественный характер взаимного поведения $\vec{x}(t)$ с близкими начальными условиями и явяляется фундаментальной характеристикой потока.

\begin{definition}[Устойчивость по Ляпунову]
	Движение является устойчивым по Ляпунову, если:
	$$\forall \varepsilon>0 \quad \exists \delta(\varepsilon) \quad \abs{x^{\prime}(0)-x(0)}<\delta(\varepsilon): \quad \forall t \quad  \abs{x^{\prime}(t)-x(t)}<\varepsilon.$$
	
	Это означает, что точки, близкие при $t=0$ остаются близкими и в последующие моменты времени.
\end{definition}

\begin{definition}[Орбитальная устойчивость (устойчвость по Пуанкаре)]
	Движение является орбитально устойчивым (устойчивым по Пуанкаре), если точки, близкие в одном месте ФП, остаются близкими и в остальным областях ФП.
\end{definition}

\begin{remark}
	Из устойчивости по Ляпунову следует устойчивость по Пуанкаре, но не наоборот.
\end{remark}

Частным случаем устойчивости по Ляпунову является асимптотическая устойчивость.

\begin{definition}[Асимптотическая устойчивость]
	Движение является асимптотически устойчивым, если точки, которые находятся на ФТр близко друг к другу в момент времени $t=0$, неограниченно сближаются по мере движения.
\end{definition}

\begin{definition}[Экспоненциальная устойчивость]
		Движение является экспоненциально устойчивым, если при этом расстояние между точками убывает не медленнее (т.е. быстрее или так же) чем $e^{-\lambda t}, \lambda>0$.
\end{definition}
 
\subsection{Задача исследования структуры ФП}

Для автономных систем с фиксированным значением параметров нахождение структуры ФП сводится в первую очередь к выделению особых точек и ФТр.

К особым траекториям и точкам относятся:

\begin{itemize}
	\item неподвижные точки (fixed point), которые находятся из условия:
			$$\vec{f}(\vec{x}, \vec{a}) \Big|_{\vec{x}=\vec{x}_{fp}}=0.$$
	\item ФТр, соответствующая периодическому движению, так как в общем случае движение системы не является периодичным. Для автономных систем такие ФТр являются замкнутыми.
	\item сепаратрисы седловых точек.
\end{itemize}

В задачу исследования структуры ФП также входит нахождение границ областей финитного движения и хаотичных компонент.

\begin{definition}[Хаотичные компоненты]
	Границы областей, в которых движение хаотично, называются хаотичными компонентами ФП.
\end{definition}

Задача исследования структуры ФП также включает в себя определение зависимости границ хаотичных компонент от значения параметров данной системы.

\begin{definition}[Финитное движение]
	Движение называется финитным, если движение системы при заданном значении параметра остается ограниченным:
	\begin{equation}\label{eq:finite_motion}
		\exists x_j>0 : \quad \forall t \quad \forall i=\overline{1,n}\quad  \abs{x_i(t)}<x_j.
	\end{equation}
\end{definition}

\begin{definition}[Инфинитное движение]
	Движение называется инфинитным, если хотя бы для одной $x_i$ утверждение (\ref{eq:finite_motion}) не выполняется.
\end{definition}

\begin{definition}[Хаос (хаотическое движение)]
	Финитное експоненциально неустойчивое по Ляпунову движение нелинейной ДС называется хаосом или хаотическим движением. 
\end{definition}

\begin{definition}[Экспоненциальная неустойчивость по Ляпунову]
	Экспоненциальная неустойчивость по Ляпунову --- это неустойчивость, при которой расстояние между близкими при $t=0$ точками ФП растет со временем по закону:
	$$\Delta(t) \sim \Delta(\sigma)e^{\sigma t},$$
	где $\sigma$ --- показатель Ляпунова.
\end{definition}

\begin{definition}[Показатель Ляпунова]
	Показатель Ляпунова --- это характерная устойчивая скорость экспоненциального расхождения близких ФТр. 
\end{definition}

\begin{definition}[Регулярное движение]
	Регулярное движение --- это нехаотическое движение. 
\end{definition}

\subsection{Задача исследования ДС}

При изменении параметра системы $\vec{a}=(a_1, a_2, \ldots, a_k)$ в общем случае изменяется и свойство её исключительных решений. Очень важным является найти границы значений параметров, при переходе через которые меняется и/или число и тип решений.

\begin{definition}[Бифуркация, точка бифуркации]
	Качественное изменение поведения ДС при переходе значений параметров системы через некоторое граничное значение называется бифуркацией, а соответствующее граничное значение $\vec{a}=\vec{a}_b$ называется точкой бифуркации.
\end{definition}

Одной из важнейших задач исследования ДС является задача исследования хаотического движения, то есть выявление областей их существования в пространстве параметров и определение зависимости характеристик таких движений (например таких, как показатель Ляпунова, корреляционные функции) от параметров ДС.

Если для регулярного движения при исследовании устойчивости важно просто знать, есть она или её нет, то для хаоса является важным исследование cамой неустойчивости.

\begin{definition}[Сценарий перехода к хаосу]
	Изменение характеристик регулярного движения при приближении к хаотическим компонентам в пространстве параметров называется сценарием перехода к хаосу.
\end{definition} 

\section{Особые точки}

Рассмотрим автономную систему:
\begin{align*}
	\frac{d\vec{x}}{dt}=f_i(\vec{x}, \vec{a}), \qquad \textnormal{где } \vec{x}&=(x_1, x_2,\ldots, x_n), \\
	\vec{a}&=(a_1, a_2, \ldots, a_k), \\
	\vec{f}&=(f_1, f_2, \ldots, f_n).
\end{align*}

Особая точка: $f_i(\vec{x}, \vec{a}) \Big|_{\vec{x}=\vec{x}_{f}}=0$.

Сами по себе особые точки не представляют интереса. Более важным является то, как ведут себя ФТр вблизи этих точек.

\begin{example}[Осцилятор без затухания]
	$$\ddot{x} + \omega^2 x = 0$$
	\begin{equation*}	
		\begin{cases}
			\frac{dx}{dt} = y \\
			\frac{dy}{dt} = -\omega^2 x
		\end{cases}	
	\end{equation*}
	
	Особая точка:
	\begin{align*}
		y &= 0, \\
		-&\omega^2 x = 0, \\
		x &= 0 \quad \Rightarrow \quad (0,0).
	\end{align*}
	
	Найдем уравнение ФТр:
	\begin{align*}
		\frac{dy}{dx} &= -\frac{\omega^2 x}{y} \\
		y d&y = -\omega^2 x dx \\
		\frac{y^2}{2} &+ \frac{x^2 \omega^2}{2} = const.
	\end{align*}
\end{example}

\begin{example}[Осцилятор с затуханием]
	$$\ddot{x} + 2 \gamma \dot{x} + \omega^2 x = 0$$
	\begin{equation*}	
		\begin{cases}
			\frac{dx}{dt} = y \\
			\frac{dy}{dt} = -2 \gamma y - \omega^2 x
		\end{cases}	
	\end{equation*}
	
	Особая точка: $(0,0)$.

	Найдем уравнение ФТр:
	\begin{align*}
		\frac{dy}{dx} &= -\frac{2 \gamma y + \omega^2 x}{y} \\
		\frac{dy}{dx} &= -2 \gamma - \omega^2 \frac{x}{y}.
	\end{align*}
	
	Рассмотрим случаи:
	\begin{itemize}
		\item $\gamma \geq 0$:
		\begin{itemize}
			\item $\gamma < \omega$: устойчивый фокус.
			\item $\gamma > \omega$: устойчивый узел.
		\end{itemize}
		
		\item $\gamma < 0$:
		\begin{itemize}
			\item $\gamma < \omega$: неустойчивый фокус.
			\item $\gamma > \omega$: неустойчивый узел.
		\end{itemize}
	\end{itemize}
\end{example}

\begin{example}
	$$\ddot{x} - r^2 x = 0$$
	\begin{equation*}	
		\begin{cases}
			\frac{dx}{dt} = y \\
			\frac{dy}{dt} =  r^2 x
		\end{cases}	
	\end{equation*}
	
	Особая точка: $(0,0)$.
	
	Найдем уравнение ФТр:
	\begin{align*}
		\frac{dy}{dx} &= \frac{r^2 x}{y} \\
		y d&y = r^2 x dx.
	\end{align*}
	
	Общий вид решения:
	$$x = c_1 e^{rt} + c_2 e^{-rt}, \quad \textnormal{где } c_1, c_2 - \textnormal{константы интегрирования}.$$
	
	Рассмотрим случаи:
	\begin{itemize}
		\item $c_1 = 0, t \rightarrow \infty \quad \Rightarrow \quad x \rightarrow 0$,
		\item $c_2 = 0, t \rightarrow -\infty \quad \Rightarrow \quad x \rightarrow 0$,	
	\end{itemize}
	
	т.е. особые точки существуют только тогда, когда либо $c_1$, либо $c_2$ равно $0$.
	
	Таким образом, уравнение ФТр:
	\begin{equation*}
		\frac{y^2}{2E} - \frac{x^2}{2E/t^2} = 1,
	\end{equation*}
	
	где $E=c_1c_2=const$.

	\begin{itemize}
		\item Если $E=0$, т.е. $\frac{y^2}{2} - \frac{x^2}{2/t^2} = E = 0$, тогда ФТр --- седло.
		\item Если $E \neq 0$, тогда ФТр --- гиперболы.	
	\end{itemize} 
\end{example}

\begin{example}[Физический маятник]
	$$\ddot{x} + \frac{q}{l} \sin x = 0$$
	\begin{equation*}	
		\begin{cases}
			\frac{dx}{dt} = y \\
			\frac{dy}{dt} = -\frac{q}{l} \sin x
		\end{cases}	
	\end{equation*}
	
	Особые точки: $y = 0; \quad x = \pm 2\pi n, \quad n \in \mathbb{Z}$.
	
	Найдем уравнение ФТр:
	\begin{align*}
		\frac{dy}{dx} &= \frac{\frac{q}{l} \sin x}{y} \\
		y d&y = \frac{q}{l} \sin x dx \\
		\frac{y^2}{2} &- \frac{q}{l} \cos x = E
	\end{align*}
\end{example}


