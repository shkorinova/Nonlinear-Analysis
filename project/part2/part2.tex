\chapter{Хаос и фракталы}

\section{Показатель Ляпунова и кинематика хаотического движения}

Случайный процесс (СП) описывается определенными математическими закономерностями (либо это стохастические уравнения движения (уравнение Ланжевина), либо уравнения для функции распределения ФР (уравнения Фоккера-Планка-Колмогорова)).

Исследование нелинейных Динамических Систем (ДС) показывает, что  для них характерны такие режимы поведения, которые обладают свойствами СП. Это называется \textit{динамическим хаосом}. Его называют также \textit{детерминированным хаосом}, так как он возникает в отсутствии каких-либо устойчивых факторов и полностью определяет начанльные условия.

Явление хаоса присуще большинству нелинейных систем (автономных и неавтономных), но оно может быть труднонаблюдаемо, если хаос является слабым или медленным, то есть проявляется только на большом промежутке $t$, либо $\exists$ на узком промежутке значений параметров.

\begin{definition}[Хаотическое движение (или хаос)]
  Финитное экспоненциальное неустойчивое движение нелинейной ДС называется хаотическим движением или хаосом.
\end{definition}

\begin{definition}[Экспоненциально неустойчивое движение по Ляпунову]
  Движение ДС называется экспоненциально неустойчивым по Ляпунову, если $\Delta(t)$ между близкими при $t=0$ ФТр растет со временем по экспоненциальному закону
  $$\Delta(t) = \Delta(0) \exp^{\Theta t}, \ \Theta > 0$$
\end{definition}

Если $\vec{x^{*}}(t)$ -- опорное движение, а $\vec{x_{1}}(t)$ -- сопутствующее движение с близкими начальными условиями, то $$\Delta(t) = \abs{\vec{x^{*}}(t) - \vec{x_{1}}(t)}$$.

А близость начальных условий означает, что $$\Delta(0) = \abs{\vec{x^{*}}(0) - \vec{x_{1}}(0)} << 1$$

Так как показатель Ляпунова в разных направлениях может быть разным, то он определяется как средняя скорость экспоненциального роста на большом интервале времени ($\Delta(t) \rightarrow \infty$), тогда:

\begin{equation}
  \Theta = \lim_{\substack{\Delta(t)\rightarrow\infty \\ \Delta(0)\rightarrow\infty}} \ln(\frac{\Delta(t)}{\Delta(0)}) \Rightarrow "\sim"
\end{equation}
\begin{equation}
  \Theta \sim \frac{1}{\tau}\ln(\frac{\Delta(t)}{\Delta(0)})\ \Rightarrow\ \tau = \frac{1}{\Theta}\ln(\frac{\Delta(t)}{\Delta(0)})
\end{equation}

Определенное время $\tau$ называется \textit{горизонтом прогноза (предсказуемости)}

В этом случае, $$\Delta(t) = \Delta(\tau)$$ является \textit{характерным размером} области движения системы в ФП.

\begin{remark}
	Система называется \textit{локально неустойчивой}, если для нее $\exists$ направление в ФП, для которого $$\Delta(t) = \Delta(0)\exp^{\Theta t}, \ \Theta>0$$
\end{remark}

\begin{remark}
  При численном интегрировании уравнения для определения $\Theta$ решается задача Коши для опорного и сопутствующего движения, тогда:
  $$\Delta(t)= \abs{\vec{x^{*}}(nt_0) - \vec{x_{n}}(nt_0)}$$ и показатель Ляпунова определяется формулой:
  $$\Theta = \frac{1}{Nt_0}\sum_{n=1}^{N}\ln(\frac{\Delta(n)}{\Delta(0)})$$
  При этом $\Delta(0)$ следует брать малым в сравнении с характерным масштабом динамических перменных, но большим по сравнению с погрешностью $\Delta_{c}$.
  Временной шаг $t_{0}$ должен быть сравним с $\Theta^{-1}$.
  Число $N$ берется достаточно большим, чтобы $\Theta$ устанавливался с нужной точностью.
\end{remark}

Характер изменения "объема" НДС зависит от выбора области ДС. ФП может содержать аттракторы (A) и репеллеры (R). В ходе эволюции ДС, имеющей А, объем области, которую она занимает, неограничено уменьшается, сжимаясь к А. Такой А часто оказывается нетривиальным множеством, движение которого стохастическое.

Это означает, что на нем:
\begin{enumerate}
  \item движение может быть локально неустойчивым
  \item движение обладает свойствами эргодичности и перемешивания
\end{enumerate}

Асимптотическая устойчивость А определяется сжатием Фазового объема:
